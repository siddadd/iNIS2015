Vision and video applications are becoming pervasive in mobile and embedded systems. 
Consumer wearable devices require capabilities for real-time video analytics 
and prolonged battery lifetimes, which is further driving 
the need for innovative system designs 
with low-power, reliability and high performance. 
Further, the increasing resolution of image sensors in these mobile systems places an increasing demand on 
both the memory storage as well as the computational power. 
Such stringent requirements have given rise to accelerator-rich architectures in system-on-chips, where the 
primary computational burden is handled by dedicated hardware accelerators.
 
In this paper we explore existing vision accelerators and analyze their architecture, performance and scalability for 
different datasets and applications. The applications evaluated in this work are neuro-biologically inspired algorithms for 
object detection, object recognition and activity recognition which are complex, compute-intensive and bandwidth-intensive. 
This paper further analyzes the reliability of such embedded vision systems in terms of robustness of performance and energy 
efficiency under different application scenarios. Specifically, this work discusses the opportunities to improve energy efficiency 
by minimizing DRAM refreshes and explores techniques to exploit algorithmic resilience for reduction in compute units
while maintaining reliable system accuracy and performance.

